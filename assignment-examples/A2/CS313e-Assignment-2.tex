\documentclass[12pt]{article}


\input{../header.tex}



\begin{document}


\title{Assignment 1}%replace X with the appropriate number
\author{CS329e - Elements of Software Design \\ %replace with your name
Infinite Spiral of Numbers \\
(100 points)\\
\red{Due Date on Canvas and Gradescope}
} %if necessary, replace with your course title

\date{\vspace{-5ex}}
%\date{}  % Toggle commenting to test

\maketitle

\begin{center}



\end{center}


\section{Description}
Consider the natural numbers laid out in a square spiral, with 1 occupying the center of the spiral.
The central 11 x 11 subset of that spiral is shown in the table below.


\begin{table}[!h]

    \begin{center}
    \begin{tabular}{|l |l |l |l |l |l |l |l |l |l |l |l |l |l | l|} \hline
    111	& 112	& 113	& 114	& 115	& 116	& 117	& 118	& 119 & 120 & 121 \\ \hline 
    110 & 73 & 74 & 75 & 76 & 77 & 78 & 79 & 80 & 81 & 82 \\ \hline 
    109 & 72 & 43 & 44 & 45 & 46 & 47 & 48 & 49 & 50 & 83 \\ \hline 
    108 & 71 & 42 & 21 & 22 & 23 & 24 & 25 & 26 & 51 & 84 \\ \hline 
    107 & 70 & 41 & 20 & 7 & 8 & 9 & 10 & 27 & 52 & 85 \\ \hline 
    106 & 69 & 40 & 19 & 6 & 1 & 2 & 11 & 28 & 53 & 86 \\ \hline 
    105 & 68 & 39 & 18 & 5 & 4 & 3 & 12 & 29 & 54 & 87 \\ \hline 
    104 & 67 & 38 & 17 & 16 & 15 & 14 & 13 & 30 & 55 & 88 \\ \hline 
    103 & 66 & 37 & 36 & 35 & 34 & 33 & 32 & 31 & 56 & 89 \\ \hline 
    102 & 65 & 64 & 63 & 62 & 61 & 60 & 59 & 58 & 57 & 90  \\ \hline 
    101 & 100 & 99 & 98 & 97 & 96 & 95 & 94 & 93 & 92 & 91  \\ \hline 
    \end{tabular} 
    \caption{Spiral of Numbers}
    \label{tab:my_label}
    \end{center}

\end{table}




This spiral has several interesting features. The southeast diagonal has several prime numbers (3, 13, 31, 57, and 91) along it. 
The southwest diagonal has a weaker concentration of prime numbers (5, 17, 37) along it.

To construct the spiral we start with 1 at the center, with 2 to the right, and 3 below it, 4 to the left, and so on. 
A part of the problem for this assignment is to figure out the 
rule to fill the spiral for an arbitrary size. Once you have that rule you can complete the rest of the assignment.



In this assignment your task is to implement a python program with the name \textbf{spiral.py}. 

\newpage 

You program should have the following input and output 

\section*{Input:}

You will read your input data from a file called spiral.in. The format of the file will be as follows:

\vspace{1cm}

\noindent 
11 \\
1 \\
42 \\ 
110 \\ 
91 \\

The first line will be the dimension of the spiral. It will always be odd and greater than 1 and less than 100. 
This will be followed by an arbitrary number of lines. There will be a single number on each line. These numbers will be numbers inside the spiral. 
Some of these numbers will be interior numbers, others will be numbers 
on the edge, and yet others will be numbers at the corners of the spirals. Assume that the input file that we will be testing your program will be valid.




\section*{Output:}


For each of the numbers inside the spiral, your output will be the sum of all the numbers adjacent to this number, but not including this number. 

For the above input file you will output on the console:

\vspace{1cm}

\noindent
44 \\ 
382 \\ 
477 \\ 
239 \\ 


We get 44 by adding the numbers adjacent to 1 (2 + 3 + 4 + 5 + 6 + 7 + 8 + 9). 
Similarly we get 239 by adding the numbers adjacent to 91 (57 + 90 + 92).

You will read your input from stdin like so:

\texttt{
\noindent
Mac: \\ 
python3 spiral.py < spiral.in
\\ \\ 
Windows: \\ 
python spiral.py < spiral.in
}

We will use our own input file to test your program. You must read the input in the format described above. 

Once you read the first line from the input file you will create a 2-D list with the spiral of numbers. Then from the 2-D list you will obtain the sum of adjacent numbers for a given number in the spiral and print it. The number of lines of input will be arbitrary and greater than 1.






\newpage

The file that you will be turning in will be called spiral.py. You will follow the standard coding conventions in Python. Here is the format of your code:

\begin{lstlisting}[language=Python]
    #  File: spiral.py
    #  Description:
    #  Student Name:
    #  Student UT EID:
    #  Partner Name:
    #  Partner UT EID:
    #  Course Name: CS 313E
    #  Unique Number: 
    #  Date Created:
    #  Date Last Modified:

    # Input: n is an odd integer between 1 and 100
    # Output: returns a 2-D list representing a spiral
    #         if n is even add one to n
    def create_spiral ( n ):
    
    # Input: spiral is a 2-D list and n is an integer
    # Output: returns an integer that is the sum of the
    #         numbers adjacent to n in the spiral
    #         if n is outside the range return 0
    def sum_adjacent_numbers (spiral, n):
    
    def main():
      # read the input file    
      # create the spiral
      # add the adjacent numbers   
      # print the result
    
    if __name__ == "__main__":
      main()
\end{lstlisting}

You may not change the names of the functions listed. They must have the functionality as given in the specifications. You can always add more functions than those listed.

For this assignment you may work with a partner. Both of you must read the paper on 
Pair Programming\footnote{Read this paper about Pair Programming \url{https://collaboration.csc.ncsu.edu/laurie/Papers/Kindergarten.PDF}} and abide by the ground 
rules as stated in that paper. 
If you are working with a partner then only one of you will be submitting the code. But make sure that your partner's name and UT EID is in the header. 
If you are working alone then remove the partner's name and eid from the header.






\input{../footer.tex}


\end{document}


