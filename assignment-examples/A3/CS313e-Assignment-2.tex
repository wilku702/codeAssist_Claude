\documentclass[12pt]{article}


\input{../header.tex}


\usepackage{caption}

\begin{document}


\title{Assignment 2}%replace X with the appropriate number
\author{CS329e - Elements of Software Design \\ %replace with your name
Collapsing Intervals\\
(100 points)\\
\red{Due Date on Canvas and Gradescope}
} %if necessary, replace with your course title

\date{\vspace{-5ex}}
%\date{}  % Toggle commenting to test

\maketitle

\begin{center}



\end{center}


\section{Description}
An interval on the number line is denoted by a pair of values like so: (3, 8). Our intervals are closed. So this interval represents all
numbers between 3 and 8 inclusive. The first number is always going to be strictly less than the second number. Normally in mathematics
we represent a closed interval with square brackets. But in our program we will represent an interval by means of a tuple and tuples
in Python are represented with parentheses. 

If we have two intervals like (7, 12) and (4, 9), they overlap. We can collapse overlapping intervals into a single interval (4, 12). But the
following two intervals (-10, -2) and (1, 5) are non-intersecting intervals and cannot be collapsed.
The aim in this assignment is take a set of intervals and collapse all the overlapping intervals and print the smallest set of non-intersecting
intervals in ascending order of the lower end of the interval and then print the intervals in increasing order of the size of the interval.


\section*{Input:}
You will need to read from standard input like so:
\vspace{0.5cm}
Mac:
\texttt{python3 intervals.py $<$ intervals.in}
Windows:
\texttt{python intervals.py $<$ intervals.in}

The first line in \textit{intervals.in} will be a single positive integer N $(1 < N < 100)$. This number denotes the number of intervals. 
This  will be followed by N lines of data. Each line will have two integer numbers denoting an interval. The first number will be strictly less than 
the second number. The intervals will not be in sorted order. Assume that the data file is correct. Here is the format of \textit{"intervals.in"}.

\begin{lstlisting}[language=Python]
15
14 17
-8 -5
26 29
-20 -15
12 15
2 3
-10 -7
25 30
2 4
-21 -16
13 18
22 27
-6 -3
3 6
-25 -14
\end{lstlisting}

You will read in each pair of numbers and create a tuple out of them. You will store the tuples in a list. After you have read all the intervals and 
the list of tuples is complete, you will call the function \textit{merge\_tuples}. This function will return a list of merged tuples in
ascending order of the lower end of the tuples.

For the input data file given the function \textit{merge\_tuples} will return:

\begin{lstlisting}[language=Python]
[(-25, -14), (-10, -3), (2, 6), (12, 18), (22, 30)]
\end{lstlisting}

You will take the output of the function \textit{merge\_tuples}  which is
a list of tuples and pass it to the function \textit{sort\_by\_interval\_size()}.
This function will return a list of tuples sorted by increasing order of the size of the intervals. If two intervals are of the same size then you 
should print the two intervals in ascending order of their lower ends. 

The list returned will be as follows:

\begin{lstlisting}[language=Python]
[(2, 6), (12, 18), (-10, -3), (22, 30), (-25, -14)]
\end{lstlisting}




\section*{Output:}
You will print your output to standard out. Your "intervals.out" will have exactly two lines. 
The first line will be the output of the function \textit{merge\_tuples} and the second line will be the output of the function \textit{sort\_by\_interval\_size()}. For the given input, the output will be as follows:


\begin{lstlisting}[language=Python]
[(-25, -14), (-10, -3), (2, 6), (12, 18), (22, 30)]
[(2, 6), (12, 18), (-10, -3), (22, 30), (-25, -14)]
\end{lstlisting}
  



\input{../footer.tex}


\end{document}
